\documentclass{article}
\usepackage{amsmath, amssymb,verbatim, amsthm, graphicx}

%exercise environment
\newcounter{exercise}
\setcounter{exercise}{0}
\newenvironment{exercise}{\addtocounter{exercise}{1} \noindent{\bf{Exercise \theexercise.}}}{\vspace{.5cm}}

\newcommand{\p}{\mathcal{P}}
\newcommand{\hide}[1]{}
\newcommand{\N}[0]{\mathbb{N}}
\newcommand{\R}[0]{\mathbb{R}}
\newcommand{\Q}{\mathbb{Q}}
\newcommand{\C}{\mathbb{C}}
\newcommand{\B}[0]{\mathbb{B^*}}
\newcommand{\Z}[0]{\mathbb{Z}}
\newcommand{\Ps}[0]{\mathbb{P}}\newcommand{\myf}[3]{{#1}\text{: }{#2} \to {#3}}
\newcommand{\norm}[1]{\left\Vert#1\right\Vert}
\newcommand{\abs}[1]{\left\vert#1\right\vert}
\newcommand{\set}[1]{\left\{#1\right\}}
\newcommand{\QED}{\hfill $\Box$}
\renewcommand{\empty}{\varnothing}
\newcommand{\mat}[4]{\left( \begin{array}{cc} #1 & #2 \\ #3 & #4  \end{array} \right)}

\textwidth 6.5in
\textheight 9in
\oddsidemargin 0in
\evensidemargin 0in
\headsep -0.5in

\setlength{\parskip}{0.25cm}

\pagenumbering{gobble}

\begin{document}


\noindent{\Large{\textsc{Modern Algebra \hfill Assignment $2$  \\ Spring 2019 \hfill Kenny Mclaren}}}

\large

\vspace{1cm}


\begin{exercise}
Construct multiplication tables for the groups $\Z_7^{\times}$, $\Z_9^{\times}$ and $\Z_{15}^{\times}$.
\end{exercise}
\\
\begin{tabular}{c|ccccccc}
$\Z_7^{\times}$  & 1 & 2 & 3 & 4 & 5 & 6\\\hline
1 & 1 & 2 & 3 & 4 & 5 & 6\\
2 & 2 & 4 & 6 & 1 & 3 & 5\\
3 & 3 & 6 & 2 & 5 & 1 & 4\\
4 & 4 & 1 & 5 & 2 & 6 & 3\\
5 & 5 & 3 & 1 & 6 & 4 & 2\\
6 & 6 & 5 & 4 & 3 & 2 & 1
\end{tabular}
\quad
\begin{tabular}{c|cccccc}

$\Z_9^{\times}$ & 1 & 2 & 4 & 5 & 7 & 8\\
\hline
1 & 1 & 2 & 4 & 5 & 7 & 8\\
2 & 2 & 4 & 8 & 1 & 5 & 7\\
4 & 4 & 8 & 7 & 2 & 1 & 5\\
5 & 5 & 1 & 2 & 7 & 8 & 4\\
7 & 7 & 5 & 1 & 8 & 4 & 2\\
8 & 8 & 7 & 5 & 4 & 2 & 1\\
\end{tabular}
\quad
\begin{tabular}{c|cccccccc}
$\Z_{15}^{\times}$ & 1 & 2 & 4 & 7 & 8 & 11 & 13 & 14 \\
\hline
1 & 1 & 2 & 4 & 7 & 8 & 11 & 13 & 14\\ 
2 & 2 & 4 & 8 & 14 & 1 & 7 & 11 & 13\\
4 & 4 & 8 & 1 & 13 & 2 & 14 & 7 & 11\\
7 & 7 & 14 & 13 & 4 & 11 & 2 & 1 & 8\\
8 & 8 & 1 & 2 & 11 & 4 & 13 & 14 & 7\\
11 & 11 & 7 & 14 & 2 & 13 & 1 & 8 & 4\\
13 & 13 & 11 & 7 & 1 & 14 & 8 & 4 & 2\\
14 & 14 & 13 & 11 & 8 & 7 & 4 & 2 & 1
\end{tabular}
\begin{exercise}
Construct a multiplication table for the dihedral group $D_4$.
\end{exercise}

\begin{tabular}{c|cccccccc}
$D_4$ & $R_0$ & $R_1$ & $R_2$ & $R_3$ & $FR_0$ & $FR_1$ & $FR_2$ & $FR_3$ \\
\hline
$R_0$ & $R_0$ & $R_1$ & $R_2$ & $R_3$ & $FR_0$ & $FR_1$ & $FR_2$ & $FR_3$\\ 
$R_1$ &$ R_1$ & $R_2$ & $R_3$ & $R_0$ & $FR_1$ & $FR_2$ & $FR_3$ & $FR_4$\\
$R_2$ & $R_2$ & $R_3$ & $R_0$ & $R_1$ & $FR_2$ & $FR_3$ & $FR_0$ & $FR_1$\\
$R_3$ & $R_3$ & $R_0$ & $R_1$ & $R_2$ & $FR_3$ & $FR_0$ & $FR_1$ & $FR_2$\\
$FR_0$ & $FR_0$ & $FR_1$ & $FR_2$ & $FR_3$ & $R_0$ & $R_3$ & $R_2$ & $R_1$\\
$FR_1$ & 
$FR_2$ & 
$FR_3$ & 
\end{tabular}

\begin{exercise}
Prove that if $G$ is a group in which every element is its own inverse, then $G$ is abelian.  
\end{exercise}

\begin{proof}[Solution.]
Pick $a,b\in G$. We know that $aa=e$ and $bb=e$, so
\[

\]
\end{proof}

\begin{exercise}
Lang, II.1.3.
\end{exercise}

\begin{proof}[Solution.]
The power set of the empty set,
\[
\p(\empty) = \{ \empty \},
\]
works.
\end{proof}


\end{document}
